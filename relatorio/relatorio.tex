\documentclass[11pt]{article}   % tipo de documento e tamanho das letras

% os seguintes pacotes estendem a funcionalidade básica:
\usepackage[a4paper, total={16cm, 24cm}]{geometry} % tamanho da pagina e do texto
\usepackage[portuguese]{babel}  % traduz para portugues
\usepackage[utf8]{inputenc}
\usepackage{graphicx}           % graficos
\usepackage{amsmath}            % matematica
\usepackage{tikz}               % diagramas
    \usetikzlibrary{shadows}
\usepackage{booktabs}           % tabelas com  melhor aspecto
\usepackage[colorlinks=true]{hyperref}           % links para partes do documento ou para a web
\usepackage{listings}           % incluir codigo
    \renewcommand\lstlistingname{Listagem}  % Listing em portugues
    \lstset{numbers=left, numberstyle=\tiny, numbersep=5pt, basicstyle=\footnotesize\ttfamily, frame=tb,rulesepcolor=\color{gray}, breaklines=true}
\usepackage{blindtext}

% -------------------------------------------------------------------------------------------
\title
{
    \includegraphics[width=0.3\textwidth]{images/logo_universidade.png}
    \\[0.1cm]
    \textbf{Códigos Ambíguos} \\
    Programação III
}

\author
{
    \textbf{Professores:} Salvador Abreu \\ Pedro Patinho \\
    \textbf{Realizado por:} Miguel de Carvalho (43108) \\ João Pereira (42864) 
}
\date{\today}

% -------------------------------------------------------------------------------------------
%                                Body                                                       %
% -------------------------------------------------------------------------------------------

\begin{document}
\maketitle

% -------------------------------------------------------------------------------------------
\section{Introdução} 


% -------------------------------------------------------------------------------------------
\section{Conclusão} % Conclusão
\hspace{0,5cm}Em suma, com a realização deste trabalho "Gerador de Árvores de Decisão" ficámos muito mais esclarecido sobre como é
feito o processo para gerar um \textbf{Árvore de Decisão}. \par
Processo este que pode ser feito com diferentes \textbf{critérios} que vão influenciar o resultado final.

As principais dificuldades foram o facto de o trabalho ter sido proposto com uma linguagem de programação no qual não nos
sentíamos à vontade, o \textbf{Python}.
Além disso foi assegurar que as funções \verb|update_data| e \verb|applySaves| cumprissem bem a sua funcionalidade, o que infelizmente não
foi possível.
% -------------------------------------------------------------------------------------------
\end{document}